%\documentclass[aps,prd,nofootinbib]{revtex4-1}
\documentclass[singlepage,notitlepage,nofootinbib,11pt]{revtex4-1}
\usepackage{amsmath}
\usepackage{graphicx}
\usepackage{subfig}
\usepackage{epsfig}
\usepackage{listings}
\usepackage[hidelinks,hyperfootnotes=false,bookmarks=false,colorlinks=true]{hyperref}

\newcommand{\eq}[1]{\begin{align*}#1\end{align*}}
\newcommand{\pmat}[1]{\begin{pmatrix}#1\end{pmatrix}}
\newcommand{\center}[1]{\begin{center}#1\end{center}}
\def\<{\langle}
\def\>{\rangle}
\def\l{\left}
\def\r{\right}


\begin{document}
\title{Problem Set 4 - G6080}
\author{Victor Genty}
\email{vgenty@nevis.columbia.edu}
\homepage{www.nevis.columbia.edu/~vgenty}
%\affiliation{Department of Physics, Duke University, Durham, NC 27707, USA}
\date{\today}
\begin{abstract}
\centering
Source code can be found at \href{https://github.com/vgenty/G6080/tree/master/ps3}{github.com/vgenty/G6080/ps4}
\end{abstract}
\maketitle
\section{Problem 1 - Monte Carlo}
The monte carlo algorithm is implemented in the previous liquid argon code with a few minor adjustments. I created a new method \verb|void LArgon::monte()| which contains the metropolis procedure.

\end{document}
