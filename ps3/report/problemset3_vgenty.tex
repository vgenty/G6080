%\documentclass[aps,prd,nofootinbib]{revtex4-1}
\documentclass[singlepage,notitlepage,nofootinbib,11pt]{revtex4-1}
\usepackage{amsmath}
\usepackage{graphicx}
\usepackage{subfig}
\usepackage{epsfig}
\usepackage{listings}
\usepackage[hidelinks,hyperfootnotes=false,bookmarks=false,colorlinks=true]{hyperref}

\newcommand{\eq}[1]{\begin{align*}#1\end{align*}}
\newcommand{\pmat}[1]{\begin{pmatrix}#1\end{pmatrix}}
\newcommand{\center}[1]{\begin{center}#1\end{center}}
\def\<{\langle}
\def\>{\rangle}
\def\l{\left}
\def\r{\right}


\begin{document}
\title{Problem Set 2 - G6080}
\author{Victor Genty}
\email{vgenty@nevis.columbia.edu}
\homepage{www.nevis.columbia.edu/~vgenty}
%\affiliation{Department of Physics, Duke University, Durham, NC 27707, USA}
\date{\today}
\begin{abstract}
\centering
Source code can be found at \href{https://github.com/vgenty/G6080/tree/master/ps3}{github.com/vgenty/G6080/ps3}
\end{abstract}
\maketitle
\section{Problem 1 - Stats of Mock Data}
\section{Problem 2 - Jackknife of Mock Data}
\subsection{1}
One
\subsection{2}
Calculation of standard deviation from native propagation of errors...\\
$\mathbf{f_1}$
\eq{
\sigma_{f_1}^2 &= \l(\frac{\partial f_1}{\partial \overline{v}_1}\r)^2\sigma_{\overline{v}_1}^2 + \l(\frac{\partial f_1}{\partial \overline{v}_2}\r)^2\sigma_{\overline{v}_2}^2,\\
&=\l(\frac{1}{\overline{v}_2}\r)^2\sigma_{\overline{v}_1}^2 + \l(\frac{\overline{v}_1}{\overline{v}_2^2}\r)^2\sigma_{\overline{v}_2}^2,
}
so
\eq{
\boxed{\sigma_{f_1}=\sqrt{\l(\frac{1}{\overline{v}_2}\r)^2\sigma_{\overline{v}_1}^2 + \l(\frac{\overline{v}_1}{\overline{v}_2^2}\r)^2\sigma_{\overline{v}_2}^2}.}
}
$\mathbf{f_2}$
\eq{\sigma_{f_2}^2 &= \l(\frac{\partial f_2}{\partial \overline{v}_3}\r)^2\sigma_{\overline{v}_3}^2 + \l(\frac{\partial f_2}{\partial \overline{v}_4}\r)^2\sigma_{\overline{v}_4}^2,\\ 
  &=\exp\l(2(\overline{v}_3-\overline{v}_4)\r)\l(\sigma_{\overline{v}_3}^2+\sigma_{\overline{v}_4}^2\r),
}
so
\eq{
\boxed{ \sigma_{f_2} = \exp\l(\overline{v}_3-\overline{v}_4\r)\sqrt{\sigma_{\overline{v}_3}^2+\sigma_{\overline{v}_4}^2}}.
}
$\mathbf{f_3}$
\eq{
\sigma_{f_3}^2 &= \l(\frac{\partial f_3}{\partial \overline{v}_1}\r)^2\sigma_{\overline{v}_1}^2 + \l(\frac{\partial f_3}{\partial \overline{v}_2}\r)^2\sigma_{\overline{v}_2}^2 + \l(\frac{\partial f_3}{\partial \overline{v}_3}\r)^2\sigma_{\overline{v}_3}^2 + \l(\frac{\partial f_3}{\partial \overline{v}_4}\r)^2\sigma_{\overline{v}_4}^2 + \l(\frac{\partial f_3}{\partial \overline{v}_5}\r)^2\sigma_{\overline{v}_5}^2,
}
so
\eq{
\boxed{\sigma_{f_3} = \sqrt{\log^2(\overline{v}_5) \left(\frac{\overline{v}_1^2 \sigma_{\overline{v}_2}^2}{\overline{v}_2^4}+\frac{\sigma_{\overline{v}_1}^2}{\overline{v}_2^2}+\frac{\overline{v}_3^2 \sigma_{\overline{v}_4}^2}{\overline{v}_4^4}+\frac{\sigma_{\overline{v}_3}^2}{\overline{v}_4^2}\right)+\frac{\sigma_{\overline{v}_5}^2 (\overline{v}_1 \overline{v}_4+\overline{v}_2 \overline{v}_3)^2}{\overline{v}_2^2\,\overline{v}_4^2\,\overline{v}_5^2}}}
}
\end{document}
