%\documentclass[aps,prd,nofootinbib]{revtex4-1}
\documentclass[singlepage,notitlepage,nofootinbib,11pt]{revtex4-1}
\usepackage{amsmath}
\usepackage{amssymb}
\usepackage{graphicx}
\usepackage{subfig}
\usepackage{epsfig}
\usepackage{listings}
\usepackage[hidelinks,hyperfootnotes=false,bookmarks=false,colorlinks=true]{hyperref}

\newcommand{\eq}[1]{\begin{align*}#1\end{align*}}
\newcommand{\pmat}[1]{\begin{pmatrix}#1\end{pmatrix}}
\newcommand{\center}[1]{\begin{center}#1\end{center}}
\def\<{\langle}
\def\>{\rangle}
\def\l{\left}
\def\r{\right}

\begin{document}
\title{Problem Set 5 - G6080}
\author{Victor Genty}
\email{vgenty@nevis.columbia.edu}
\homepage{www.nevis.columbia.edu/~vgenty}
\date{\today}
\begin{abstract}
\centering
Source code can be found at \href{https://github.com/vgenty/G6080/tree/master/final}{github.com/vgenty/G6080/final}
\end{abstract}
\maketitle
\section*{Introduction}
Blah
\section*{1.1 Setup}
\begin{enumerate}
\item Te updating scheme in Eq. (2) is unitary because it preserves the integral squared norm of the state vector in time i.e.
  \begin{align*}
    \int_{\mathbb{R}} dx \left|\phi(t+\Delta t)\right|^2 = \int_{\mathbb{R}} dx \left|\phi(t)\right|^2.
  \end{align*}
  Therefore it's sufficient to check if the updating scheme is norm preserving at $n+1$ and $n$. So,
  \begin{align*}
    \left|\phi^{(n+1)}\right|^2 &= \left|\frac{1-i H \Delta t/2}{1+i H \Delta t/2}\right|\left|\phi^{n}\right|^2 \\
    &= \frac{1-i H \Delta t/2}{1+i H \Delta t/2}\cdot\frac{1+i H^{\dagger} \Delta t/2}{1-i H^{\dagger} \Delta t/2}.
  \end{align*}
  Now the Hamiltonian is hermitian, $H = H^{\dagger}$ so,
  \begin{align*}
    \frac{1-i H \Delta t/2}{1+i H \Delta t/2}\cdot\frac{1+i H \Delta t/2}{1-i H \Delta t/2} &=  1,\\
  \end{align*}
  as the numerator and denominator cancel leaving,
  \begin{align*}
    \left|\phi^{n+1}\right|^2 = \left|\phi^n\right|^2.
  \end{align*}
  Therefore the updaing scheme is unitary.
\item The Schrodinger equation and be nondimensionalized with the substitutions,
  \begin{align*}
    \widetilde{\phi}&=\phi\sqrt{\frac{\hbar}{m c}}\\
    \widetilde{x}&=x\left(\frac{mc}{\hbar\right)}\\
    \widetilde{t}&=t\left(\frac{mc^2}{\hbar}\right)\\
    \widetilde{V}&=V\left(\frac{1}{mc^2}\right),
  \end{align*}
  leaving the Schrodinger equation as,
  \begin{align*}
    -\frac{1}{2}\frac{\partial^2}{\partial\widetilde{x}^2}\widetilde{\phi}+\widetilde{V}\widetilde{\phi} = i \frac{\partial}{\partial\widetilde{t}}\widetilde{\psi}.
  \end{align*}
  This gives the updating scheme the following form,
  \eq{
    \chi_{j+1}^{(n)}+\left(-2+\frac{4i\widetilde{\epsilon}^2}{\Delta\widetilde{t}}-2\widetilde{\epsilon}^2\widetilde{V_j}\right)\chi_j^{(n)} + \chi_{j-1}^{(n)} = \frac{8i\widetilde{\epsilon}^2}{\Delta\widetilde{t}}\phi_j^{(n)}.
  }
  The total range of the simulation is fixed to $-1 < x < 1$. We also set the mass $m$ equal to 1. {\bf come back to me}
\end{enumerate}
\section*{1.2 Evolution of a wavepacket}
We start with a normalized gaussian wave packet at $t=0$,
\eq{
\phi(x,t=0) = \frac{1}{\sqrt{\sigma \sqrt{\pi}}}\exp\left(ikx\right)\exp\left(-(x-x_0)^2/\left(2\sigma^2\right)\right).
}
This particular wave packet has a fixed expecation energy value given by,
\begin{align*}
  \<E\> &= \<\phi|E|\phi\> = \<\phi|\frac{p^2}{2m}|\phi\> = \int_{\mathbb{R}}dx\<\phi|x\>-\frac{\hbar^2}{2m}\frac{\partial^2}{\partial x^2}\<x|\phi\> \\
  &= \frac{p^2}{2m} + \frac{\hbar^2}{2 m \sigma^2},
\end{align*}
and with our choice of units is
\eq{
\<E\> = \frac{k^2}{2} +  \frac{1}{4\sigma^2}.
}
Rearranging we can choose the input momentum based on the given energy,
\eq{
k = \sqrt{2 E - \frac{1}{2\sigma^2}}
  }
\end{document}
